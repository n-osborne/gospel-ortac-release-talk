\documentclass[pdf]{beamer}
\mode<presentation>{}

\beamertemplatenavigationsymbolsempty
\setbeamertemplate{itemize items}[circle]
\setbeamertemplate{enumerate items}[circle]

\usepackage[utf8]{inputenc}
\usepackage{hyperref}

\title{Ortac/QCheck-STM 0.1, Gospel 0.2}
\author{
  Samuel Hym
  \and
  Nicolas Osborne
}
\titlegraphic{
  \includegraphics[width=20mm]{./img/tarides.png}
}
\date{October 20, 2023}

\begin{document}

\begin{frame}
  \titlepage
\end{frame}

% Presentation outline
\begin{frame}{Outline}
    \tableofcontents[hideallsubsections]
\end{frame}

% Current section
\AtBeginSection[ ]
{
\begin{frame}{Outline}
    \tableofcontents[currentsection]
\end{frame}
}

\section{Ortac/QCheck-STM 0.1}

\begin{frame}{What is QCheck-STM?}

  \texttt{QCheck-STM} is a model-based testing framework that builds upon
  \texttt{QCheck}. According to a library description, it generates random
  programs using the functionalities of this library and runs them, records
  the results at each step of the run, and compares these results with the
  behaviour of a given pure (functional) model.

\end{frame}

\begin{frame}

  The library description contains mostly:

  \begin{itemize}

    \item a \texttt{cmd} type encoding the tested functions with their types
    \item how to instantiate the type we are testing (called \texttt{sut})
    \item a \texttt{state} type (the model)
    \item a \texttt{next\_state} function
    \item predicates for pre- and post-conditions

  \end{itemize}

\end{frame}

\begin{frame}{Towards a \texttt{Gospel} to \texttt{QCheck-STM} translation}

  The general context is to aim for a unified push-button experience for
  testing frameworks in the property based testing familly (\texttt{Crowbar},
  \texttt{QCheck}, \texttt{QCheck-STM}\dots).

  Specify once (with \texttt{Gospel}) and have property based testing ---
  with or without fuzzing --- and model-based testing for free!

\end{frame}

\begin{frame}{The Proof of Concept}

  Naomi Spargo build a Proof of Concept \url{https://github.com/naomiiiiiiiii/stm_ortac}.

  The general basic idea is to build on Gospel's models to implement the
  \texttt{state} type and postconditions for the \texttt{next\_state} function.

\end{frame}

\begin{frame}{The new \texttt{ortac}}

  \begin{itemize}
    \item reorganisation of \texttt{ortac} with a plugin architecture
      \begin{itemize}
        \item less dependencies
        \item fine-grain release
        \item \texttt{ortac-core} package exposes an (use-)independant
          \texttt{ocaml-of-gospel} translation
      \end{itemize}
    \item complete rewrite of Naomi's PoC
  \end{itemize}

\end{frame}

\begin{frame}
  \centering
  \huge{Demo}
\end{frame}

\section{Gospel 0.2}

\begin{frame}{A new \texttt{gospel.ppx}}

  \begin{itemize}
    \item Build documentation attributes based on the contents of the gospel ones
    \item For instance: \url{https://ocaml-gospel.github.io/gospel/gospel/Gospelstdlib/index.html}
    \item One can also generate and install locally manpages for Gospelstdlib.
  \end{itemize}

\end{frame}

\begin{frame}{What's new in Gospel 0.2?}

  \begin{itemize}
    \item Pattern matching exhaustiveness and redundancy analysis (thanks to Paul Patault)
    \item Bugfixes and performance improvements of the pps
    \item A new \texttt{dumpast} subcommand
    \item Partial application allowed in Gospel terms
    \item Tuple destruction in anonymous functions
    \item Better location tracking and error messages
    \item Updated documentation (with typechecking of Gospel extracts)
  \end{itemize}

\end{frame}

\section{Future work}

\begin{frame}{Future work}

  \begin{itemize}
    \item battle test and improve the \texttt{qcheck-stm} plugin
    \item dune support for Gospel
  \end{itemize}

\end{frame}

\end{document}
